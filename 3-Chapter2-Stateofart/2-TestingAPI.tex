\section{Testing in RESTful API}\label{sec:API-testing}

The software tests on RESTful APIs\cite{Golmohammadi2023} evaluate different aspects of information quality and security metrics. Strategies for testing include regression\cite{Godefroid2020}, unit tests\cite{Golmohammadi2023}, White-Box\cite{Arcuri2021}, property based\cite{Karlsson2020}, and others.

To apply the tests on RESTFul API\cite{De2017}, the actions to be performed must be evaluated, including checking that the HTTP status codes are correct (e.g verifying that the response for forbidden access is correct), verifying that the payload is in the correct format, and confirming that the responses are providen in a reasonable time. Additionally, different scenarios such as the happy path, negative situations with correct and incorrect input and security in terms of data encryption, data security and the correct configuration of access permissions must be considered.

Several challenges have been identified in these tests\cite{Ehsan2022}, among which is the generation of a framework to generate effective unit tests to validate the quality of this type of applications. Furthermore, limitations have been found corresponding to the description of how a RESTful API works, since there are several standards, among which are XML, JSON or OpenAPI.  Some research aims to solve these challenges, as in the case of unit test generation\cite{Arcuri2019,segura2018,Arcuri2017,Viglianisi2020} and improving the test coverage\cite{Wu2022}.
