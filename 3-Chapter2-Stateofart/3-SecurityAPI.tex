\section{Security in RESTful-API}\label{sec:security}

 The OpenAPI Specification(OAS)\footnote{https://swagger.io/specification/, September 2023} provides an implementation-independent specification on how APIs should work. However, potential vulnerabilities have been identified, such as identifying sensitive information in data, detecting the level of risk that using RESTFul APIs could pose in the transmission of sensitive information, and determine the level of exposure of sensitive data. As of 2021 a literature identified these risks as significant challenges for information security\cite{Sun}.

Many techniques are currently used in RESTFul API security\cite{Siriwardena2020}. These include data encryption with TLS, authentication with the OAuth protocol, the use of security certificates, specification of roles to restrict access to data, decentralization for different access roles, and software patterns that delegate information to different users, thereby facilitating control of the information exchanged.

The OWASP API Security Risks is an important standard for developing applications. Several studies have considered the implications of the creation of Web applications\cite{Muhammad2021,Idris_Syarif_Winarno_2022,Cheh2021} and have found that issues realted to proper authorization to access data, differentiating sensitive data from non-sensitive data, and evaluating how the calls made to RESTFul APIs are applied.

The most common forms of attack\cite{Modi2022} are the Structured Query Language (SQL) Injection Attack, a Man-in-the-Middle Attack (MitM), phishing attack, Denial of Service Attack (Dos). Thee types of attacks can represent a risk to the information handled by the RESTFul API. Statistics indicate an increasing number of attacks targeting these types of applications.

Developers have created some techniques\cite{Munsch2021,zenodo2023} to mitigate the risks related to unauthorized access to information, including the use of secure protocols such as SSL, use of hashing algorithms, and application of access rules and user roles. These strategies should be validated in the respective software tests.

Therefore, organizations must consider security as a fundamental pillar in the development of this type of applications Many vulnerabilities are not the result of attacks but from omissions by software development team or of the tools used for its coding. Considering that RESTful APIs are the most commonly used mechanism for information exchange between applications, they represent a critical component that must be prioritized.

In recent years, different techniques have been proposed for the evaluation of security in RESTFul APIs, including automatic black box testing\cite{Corradini2023}, automatic penetration testing\cite{Auricchio2022}, model-based\cite{EMEKA2023} and using machine learning\cite{Ghanem2019,Hu2020,Schwartz2019}. These security testing methods allow to automate its design and execution to improve the coverage of possible vulnerabilities that may have been generated during the development process or through the interaction of different components. Other technique is the generation of vulnerability samples using machine learning algorithms\cite{Nong2022GeneratingRV}, it allows automating the process of generate test cases for evaluating vulnerabilities. This approaches faces challenges about the data quality to generate effectively vulnerability samples. In contrast, using machine learning and deep learning algorithms can improve the vulnerability sampling and detection accuracy.

Some challenges have been identified in securing RESTful APIs\cite{zhong2023}, including handling interactions with third-party components, ensuring good development practices such as emerging development styles, applying appropriate access configurations, and testing data access using white-box testing. The latter is crucial because black-box testing alone is insufficient to identify all vulnerabilities.
