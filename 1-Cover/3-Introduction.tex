\chapter*{Introduction}

Mutation testing is a set of practices that consists in the application of mutation operators, which are the introduction of changes into the applications code and see if the security tests can detect them. These are useful in the security analysis of policies\cite{8967426}, fail models\cite{6569774} and quality of software updates\cite{kravets2012feasibility}.

A RESTful API Cloud Service uses HTTP requests to exchange data, enabling operations of reading, updating, creating and deleting across data, hence, the security of the exchange of the data is an important concern in software development.

There are some risks according OWASP foundation related to authorization, authentication, resource consumption and encryption of the data, one of the most important is the configuration of the security policies, because a bad configuration can open the gate to unauthorized access.

Security of the exchange of data in cloud services is an enormous challenge because it is essential to protect the information involved of potential intruders. Some services are based in the integration between different sources of data handled through RESTFul APIs, hence developers must apply different security tests in development cycles to ensure the security polices are correct and the information is safe.

There are challenges related to the security of the information that is exchanged in the RESTFul APIs, because unknown vulnerabilities are revealed constantly because of situations that arise with unauthorized access to sensitive information. Software testing evaluates vulnerabilities, but there are limitations regarding certain situations that cannot be evidenced immediately, hence security-aware mutation operators can be designed that can provide an improvement for the detection of these through mutation testing.

The goal of this research proposal is to extend the work of Loise Thomas, \textit{Towards secure-aware mutation Testing} \cite{Loise2017}, which designed mutation operators to test security policies in Java. This problem is similar to the configuration of security policies in RESTFul API Cloud Services, because through them it is possible to control de access and the privileges of their users.

In this way, the state of the art shows that exists a limitation in the mutation operators because these only works in limited backgrounds because their design is only for a specific language or library. Thus, it reinforced the idea about a potential contribution of a set of secure-aware mutation operators for the application of new and better strategies to test the security of the RESTFul API.

The structure of this document is as follows: Chapter 2 presents the research problem and the objectives. Chapter 3 discusses the challenges in mutation testing and security in RESTful APIs. Chapter 4 explains the methodology used in this work. Finally, the expected results are described in Chapter 5.
