\chapter*{Abstract}
% [VA]: Just for you to read
%
% Mutation-based fuzzing is a test input generation technique where an initial set of well-formed inputs, called seeds, is
% used as a basis to perform alternations, called mutations, which in turn leads to new test inputs, called mutants
% Usually, the goal is to use fuzzing in order to produce test input that will detect security issues and unexpected flaws.

%Background
Today, Cloud Services are predominantly used through Application Programming Interfaces (APIs) that conform to the
REpresentational State Transfer (REST) software architectural style --- we call such services RESTful cloud services.
As any piece of software, cloud services also have bugs with reliability and security implications, since
users of these cloud services are likely to share sensitive information while interacting with them.

Research suggests that current security tools are not enough to cover privacy problems and vulnerabilities in RESTful cloud services, some of these problems are based on an unappropriated security configuration.
The goal of this proposal is to investigate and evaluate the mutation of the security configuration though the use of security-aware mutations in RESTful cloud services in order to eventually ensure the correct configuration of the security policies of these services.

%methods
Our proposal aims to study specific vulnerabilities derivated of the security configuration in Flask and Django Model-View-Controler (MVC) frameworks, used in
Python language to build RESTful cloud services, within the set of OWASP Top-10 2023 vulnerabilities, 
and identify gaps in the effectiveness of the tests typically performed during software development cycle.
Then, a testing using security-aware mutations that can be used to automate testing the quality of the security configuration of RESTful API services.

%hypotehsis
