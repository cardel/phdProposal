\section{Context}

RESTful API is an architectural style for designing web services. It is based on the principle of using HTTP requests to access resources, making it a very flexible and scalable way to exchange information between systems. However, API-Restful also introduces some security challenges.

One of the main challenges is the applying of security policies. These services exchange sensitive data, such as passwords, credit card numbers, personal information, which can be easily intercepted by attackers. Additionally, Restful API is often used to access resources that are not protected by authentication and authorization mechanisms. This vulnerability means that attackers can easily gain unauthorized access to sensitive data.

To address these security challenges, modern practices include encrypting communication, requiring authentication to access data, implementing input validation mechanisms, and restricting access to the resources.

\section{The problem}

The security of the information in RESTful APIs is a critical concern, particularly because some of them exchange sensitive information and private data. Mechanisms such as authorization, access policies, access restrictions and encryption are employed to protect the information. Therefore, developers must create mechanisms to configure and secure the information against unauthorized access.

According to the open Worldwide Application Security Project (OWASP) in 2023, there is an increase in the security risks in the operation of RESTful API, such as, authorization lacking, uncontrolled resource consumption, security misconfiguration, and unauthorized access to data. Companies need to invest significant effort in updating applications, improving security policies, and monitoring the exchange of data in RESTFul API. This includes using of encrypt protocols like HTTPS/TLS, implementing authorization mechanisms such as OAuth, monitoring activities of data exchange, applying some access restrictions like Cross-Origin-Resource-Sharing (CORS) and applying appropriate security policies\cite{Riggs2023}.

The literature revise shows that over the past decade, the importance of detecting and avoiding vulnerabilities through software testing\cite{8564344} during the development cycle has grown significantly.  This is because software testing can help identify and fix security vulnerabilities early in the development process, before they can be exploited by attackers. Recent research indicate that the risks of security of APIs are a major concern. Some critical vulnerabilities include broken object-level authorization, which occurs when an attacker can access resources breaking the authorization mechanisms; broken user authentication, where an attacker gains unauthorized access to a RESTful API by bypassing the authentication mechanism, and excessive data exposure, when too much sensitive data is exposed to the public, making it impossible to control effectively control access.

Other studies about API security risks\cite{zenodo}, injection attacks, these attacks involve injecting malicious code into an API. Rate limiting attacks involve sending too many requests to an API in a short period of time, which can overload the API and cause it to crash. Denial-of-service attacks: These attacks involve flooding an API with requests, which prevents legitimate users from accessing the API. Developers need to be aware of these risks and take steps to mitigate them, some of them are related to using encrypt algorithms.

Mutation testing is an useful tool to evaluate the capacity of the security tests to detect vulnerabilities and allow to study how to detect potential new vulnerabilities in the software, because mutation testing can be used to create new scenarios that the test suite has not yet encountered. By mutating the code and evaluating if the test process can detect the introduced vulnerabilities, mutation testing can help to identify potential new vulnerabilities that may not have been found by other testing methods.

The strong potential of mutation testing lies in its to ability to detect unexpected vulnerabilities for developers, as mutation operators can simulate risks situations exploited by intruders. Therefore, the specification of RESTful APIs operators security-aware mutation operators can provide a new framework of security tests for the evaluation of the quality of the security tests performed by developers.

\section{Research question}

  How can security-aware mutation operators be designed to improve the coverage of security testing for vulnerabilities in the configuration of security policies in RESTful APIs?
