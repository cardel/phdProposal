\section{Context}

API-Restful Cloud Service is an architectural style for designing web services. It is based on the principle of using HTTP requests to access resources. This makes it a very flexible and scalable way to exchange information between systems. However, API-Restful also introduces some security challenges.

One of the main challenges is the applying of security policies because these services exchange data, which can be sensitive such as passwords, credit card numbers, personal data and can be easily intercepted by attackers. Another challenge is that API-Restful is often used to access resources that are not protected by authentication and authorization mechanisms. This means that attackers can easily gain unauthorized access to sensitive data.

Nowadays, to address these security challenges, the communication is encrypted, it is necessary to authenticate to access to the data and mechanisms of validation input data and restrict the access to the API.


\section{The problem}

Security of the information in RESTful APIs is a sensitive problem because some of them exchange sensitive information and private date, there are some mechanisms to protect the information like authorization, policies of access and encryption, but it is a hard task to block the access of intruders because every day new vulnerabilities appear.  Therefore, developers must create mechanisms to configure and secure the information against unauthorized access.

According to the open Worldwide Application Security Project (OWASP) in 2023 there is an increase in the security risks in the operation of API-Restful related to authorization, resource consumption, security misconfiguration, and unauthorized access to data. Nowadays, companies need to invest significant effort in updating applications, improving security policies, and monitoring the exchange of data in APIs. Nowadays companies need to invest an important effort to up-to-date the applications, improving the security policies and monitoring the exchange of date in APIs, such ash using of encrypt protocols like HTTPS/TLS, authorization mechanisms like OAuth, monitor the activity of the APIs, applying some restrictions of access like Cross-Origin-Resource-Sharing (CORS) and appliyng some policies about their use\cite{Riggs2023}.

The literature revise shows that during the decade has growth the importance of detecting and avoiding vulnerabilities through software testing\cite{8564344} during the development cycle has grown.  This is because software testing can help to identify and fix security vulnerabilities early in the development process, before they can be exploited by attackers. Recent research has shown that the risks of security of APIs are still a major concern. A study by [2] found that the top three API security risks in 2023. The most important is broken object-level authorization, this occurs when an attacker is able to access resources that they should not have access to. Other important risk is the broken user authentication, where an attacker is able to gain unauthorized access to an API by bypassing the authentication mechanism. Third, excessive data exposure, when too much sensitive data is exposed to the public and it is not possible to control effectively the access to them..

Other studies about API security risks\cite{zenodo}, injection attacks, these attacks involve injecting malicious code into an API. Rate limiting attacks involve sending too many requests to an API in a short period of time, which can overload the API and cause it to crash. Denial-of-service attacks: These attacks involve flooding an API with requests, which prevents legitimate users from accessing the API. Developers need to be aware of these risks and take steps to mitigate them, some of them are related to using encrypt algorithms.

Mutation testing is an useful tool to evaluate the capacity of the security tests to detect vulnerabilities and allow to study how to detect potential new vulnerabilities in the software, because mutation testing can be used to create new scenarios that the test suite has not yet encountered. By mutating the code and evaluating if the test process can detect the introduced vulnerabilities, mutation testing can help to identify potential new vulnerabilities that may not have been found by other testing methods.

The strong potential of mutation testing is the detection of unexpected vulnerabilities for the developers because mutation operators can simulate risks situations which are exploited by intruders. Therefore, the specification of RESTful operators secure-aware mutation operators can provide a new framework of security tests for the evaluation of the quality of the protection of the data.

\section{Research question}

¿How to design fuzzed secure-aware mutation operators in the coverage of vulnerabilities in exchanging data e in RESTful APis?
